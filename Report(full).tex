%% -*- coding: utf-8 -*-
\documentclass[12pt,a4paper]{scrartcl} 
\usepackage[english,russian]{babel}
\usepackage[utf8]{inputenc}
\usepackage{indentfirst}
\usepackage{misccorr}
\usepackage{graphicx}
\usepackage{amsmath}
\begin{document}
	\begin{titlepage}
		\begin{center}
			\large
			МИНИСТЕРСТВО НАУКИ И ВЫСШЕГО ОБРАЗОВАНИЯ РОССИЙСКОЙ ФЕДЕРАЦИИ
			
			Федеральное государственное бюджетное образовательное учреждение высшего образования
			
			\textbf{АДЫГЕЙСКИЙ ГОСУДАРСТВЕННЫЙ УНИВЕРСИТЕТ}
			\vspace{0.25cm}
			
			Инженерно-физический факультет
			
			Кафедра автоматизированных систем обработки информации и управления
			\vfill

			\vfill
			
			\textsc{Отчет по практике}\\[5mm]
			
			{\LARGE \textit{Написать калькулятор – аналог стандартного калькулятора Windows.}}
			\bigskip
			
			1 курс, группа 1УТС
		\end{center}
		\vfill
		
		\newlength{\ML}
		\settowidth{\ML}{«\underline{\hspace{0.7cm}}» \underline{\hspace{2cm}}}
		\hfill\begin{minipage}{0.5\textwidth}
			Выполнил:\\
			\underline{\hspace{\ML}} К.\,В. Седой\\
			«\underline{\hspace{0.7cm}}» \underline{\hspace{2cm}} 2021 г.
		\end{minipage}%
		\bigskip
		
		\hfill\begin{minipage}{0.5\textwidth}
			Руководитель:\\
			\underline{\hspace{\ML}} С.\,В. Теплоухов\\
			«\underline{\hspace{0.7cm}}» \underline{\hspace{2cm}} 2021 г.
		\end{minipage}%
		\vfill
		
		\begin{center}
			Майкоп, 2021 г.
		\end{center}
	\end{titlepage}


\section{Текстовая формулировка задачи}
\label{sec:exp:code}
\begin{verbatim}
Написать калькулятор (четыре арифметических операции с возможностью их запоминания) 
– аналог стандартного калькулятора Windows.
Алгоритм:
1)	Задается число
2)	Вводится операция
3)	Вводится следующее число
4)	Так до тех пор, пока не будет введена команда очистки (например, буква с) или пока программа не завершит работу.

\end{verbatim}
\section{Код приложения}
\label{sec:exp:code}
\begin{verbatim}
#include<Windows.h>
#include<iostream>
#include<stdlib.h>
using namespace std;
bool clear = false, END = false; int i = 0;
int translator(char symbol) {
	symbol = (int)symbol;
	return symbol;
}
float algorithm(int symbol, float first_number, float second_number) {
	switch (symbol) {
	case '+':
		first_number += second_number;
		return first_number;
		break;
	case '-':
		first_number -= second_number;
		return first_number;
		break;
	case '*':
		first_number *= second_number;
		return first_number;
		break;
	case '/':
		first_number /= second_number;
		return first_number;
		break;
	case 'c':
		cin >> first_number;
		return first_number;
		break;
	case 'C':
		cin >> first_number;
		return first_number;
		break;
	case 'с':
		cin >> first_number;
		return first_number;
		break;
	case 'С':
		cin >> first_number;
		return first_number;
		break;
	default:
		return first_number;
	}
}
int main() {
	setlocale(LC_ALL, "Rus");
	float first_number = 0, second_number; int numeric_symbol; char symbol; bool sequence = false;
		for (; !END; i++) {
			if (i == 0)
				cin >> first_number;
			cin >> symbol;
			numeric_symbol=translator(symbol);
			if ((numeric_symbol != 69) && (numeric_symbol != 101) && (numeric_symbol != 133) && (numeric_symbol != 165)) {
				if ((numeric_symbol != 67) && (numeric_symbol != 99) && (numeric_symbol != 145) && (numeric_symbol != 225)) {
					cin >> second_number;
				}
				else {
					first_number = 0;
					second_number = 0;
					sequence = true;
					cout << "Cleared" << endl;
				}
				first_number = algorithm(symbol, first_number, second_number);
				if(sequence == false)
					cout << "Ответ:" << first_number << endl << first_number;
				sequence = false;
			}
			else {
				END = true;
				cout << "Algorithm was ended"<<endl;
				break;
			}
		}
	return 0;
}
\end{verbatim}
\section{Скриншоты программы}
\label{sec:picexample}
\begin{figure}[h]
	\includegraphics[width=0.9\textwidth]{Latex/1.png}
	\caption{Пример работы калькулятора}\label{fig:par}
\end{figure}

\begin{thebibliography}{9}
\bibitem{Knuth-2003}Кнут Д.Э. Всё про \TeX. \newblock --- Москва: Изд. Вильямс, 2003 г. 550~с.
\bibitem{Lvovsky-2003}Львовский С.М. Набор и верстка в системе \LaTeX{}. \newblock --- 3-е издание, исправленное и дополненное, 2003 г.
\bibitem{Voroncov-2005}Воронцов К.В. \LaTeX{} в примерах. 2005 г.
\end{thebibliography}
\end{document}
